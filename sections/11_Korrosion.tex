\section{Korrosion}
Definition: Reaktion eines (metallischen) Werkstoffes mit seiner Umgebung, die zu messbaren Veränderungen des Werkstoffs und zu Korrosionsschäden führen kann. \\

Merkmale einer Metallkorrosion: Metall reagiert immer als RM ($e^-$-Spender). Es muss immer ein OM vorhanden sein (meist $O_2, H_2O$). Korrosion ist immer möglich wenn $\Delta G  < 0$. \\

Chemische Korrosion: Ox. und Red. finden am gleichen Ort statt (ohne Elektrolyt). Bsp. Verzunderung von Stahl. \\

Elektrochemische Korrosion: Ox. und Red. sind räumlich getrennt $\Rightarrow$ Bildung einer galvanischen Zelle mit wässrigem Elektrolyt (Wasser). \\

\subsection{$O_2$-Typ}
Oxidationsmittel ist Sauerstoff $O_2$. Reaktionsgeschwindigkeit relativ klein. \\
$O_2 + H_2O + 4 e^- \leftrightarrow 4 OH^-$ (Potential von pH-Wert abhängig)\\

\subsection{$H_2$-Typ}
Oxidationsmittel ist $H^{+I}$. Reaktionsgeschwindigkeit relativ gross, weil $H_2$-Gas entweicht.  \\

Reduktionsreaktion ist vom pH-Wert abhängig: \\
sauer: $2 H^+ + 2 e^- \leftrightarrow H_2$ \\ resp. $2 H_3O^+ + 2 e^- \leftrightarrow H_2 + 2 H_2O$ \\
neutral-basisch: $2 H_2O + 2 e^- \leftrightarrow H_2 + 2 OH^-$ \\

\subsection{(Passive) Oxidschichten}
Alle Metalle ausser Au und Platinmetalle (Ru,Rh,Pd,Os,Ir,Pt) reagieren spontan mit $O_2$ unter Ausbildung einer dünnen Oxidschicht. \\

Bsp. $2Fe + \frac{3}{2} O_{2(g)} \rightarrow Fe_2O_3$ mit $\Delta G < 0$. (Dicke der Oxidschicht: nm) \\

Definition Passivoxidschicht: Oxidschicht, welche kompakt und fest haftend ist und somit das darunter liegende Metall vor weiterer Oxidation schützt (Anodischer Korrosionsschutz). \\

Pilling-Bedworth-Verhältnis (PBV): charakterisiert Oxidschicht.
\begin{equation*}
	PBV = \frac{\text{Volumen(Metalloxid)}}{\text{Volumen(Metall vor Oxidation)}}
\end{equation*}
\begin{table}[htbp]
	\begin{tabular}{ll}
		PBV $<$ 1 & rissige, nicht schützende Oxidschicht \\ & Bsp. Mg (0.8), Na (0.3)\\
		PBV = 1..2 & kompakte, schützende Passivoxidschicht \\ & Bsp. Al (1.3), Fe (2.1), Ni, Cu\\
		PBV $\gg$ 2 & abblätternde, nicht schützende Schicht \\ & Bsp. V(3.2), W (3.4), Rost (3.6) \\
	\end{tabular}
\end{table}

\subsection{Korrosion in wässrigen Lösungen}
Voraussetzung: Metall hat Kontakt zu einer Elektrolytflüssigkeit. 

\begin{enumerate}
	\item Depassivierung: (teilweise) Zerstörung des Oxidfilms
	\item Eigentliche Korrosion (Redox-Reaktion): \\
		Oxidation: $Me \rightarrow Me^{z+} + z e^-$ \\
		Reduktion: abhängig von OM und pH-Wert ($H_2$-Typ oder $O_2$-Typ)
\end{enumerate}

\subsection{Passivatoren}
Passivatoren (Inhibitoren) sind Stoffzusätze im Metall oder in der Elektrolytlösung, die den Passivoxidfilm stabilisieren. \\

Passivatoren vergrössern die Aktivierungsenergie $E_a$ der Anodenreaktion und bewirken somit einen anodischen Schutz. \\

Bsp. Eisen (Fe): Hydroxid bei $pH>12$, Chromat ($CrO_4^{2-}$),  ... \\
Bsp. Aluminium (Al): Nitrat ($NO_3^-$) \\

\subsection{Depassivatoren}
Depassivatoren (Katalysatoren) sind gelöste Stoffe, die den Passivoxidfilm lokal zerstören und damit sie die Aktivierungsenergie senken. Bsp. Chlorid ($Cl^-$) oder Säure ($H_3O^+$) für Eisen und Aluminium.

\subsection{Erscheinungsformen der Korrosion}

\subsubsection{Gleichmässige Flächenkorrosion}
Auflösung des Werkstoffs gleichmässig über die gesamte Metalloberfläche verteilt, Geschwindigkeit überall gleich gross. Bei ausreichender Materialdicke relativ ungefährlich.

\subsubsection{Kontaktkorrosion (Bimetallkorrosion)}
Korrosion eines unedleren Metalls, das mit einem edleren Metall elektrisch und via wässrigem Elektrolyt verbunden ist. Dabei bildet das edlere Metall die Kathode (und ist kathodisch geschützt), während das unedlere Metall die Opferanode bildet. \\

Massnahmen zur Vermeidung: Elektrische Isolation, Wahl kompatibler Werkstoffe, Verhinderung von Feuchtigkeit, Beschichtung der Kathode oder von Anode und Kathode. \\

\subsubsection{Lochfrasskorrosion}
Stark lokalisierte Korrosion, die zur Bildung enger, tiefer Löcher führt. Lochfrasskorrosion ist gefährlich, da sie kaum erkennbar ist und innert kurzer Zeit zur Durchlöcherung führen kann.

\subsubsection{Belüftungselemente bei passivierbaren Metallen}
Korrosion infolge räumlich variierendem $O_2$-Gehalt im Elektrolyten. An Stellen mit wenig $O_2$ kann die Passivoxidschicht nicht erneuert werden (lokale Depassivierung). Dies kann zu Lochfrasskorrosion führen. 

\subsubsection{Spannungsrisskorrosion (SpRK)}
Ein unter Spannung stehender Werkstoff reisst nach Einwirkung deines korrosiven Mediums. Charakteristik: Rissbildung senkrecht zur Spannungsrichtung. Bsp. Dampfkessel. \\

Bedingungen: Zugspannung, empfindlicher passivierter Werkstoff, Spezifisch angreifendes Korrosionsmittel. \\
Vermeidung: Vermeiden hoher Zugspannung, Entfernen von schädlichen Komponenten im Elektrolyt, Kathodischer Schutz oder Beschichtung.

