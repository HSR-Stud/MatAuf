\section{Molekulare Stoffe}
Moleküle sind abgeschlossene Atomverbände aus \emph{nichtmetallischen} Atomen. 

Atombindung (kovalente Bindung): Gemeinsames bindendes Elektronenpaar zwischen 2 Nichtmetallatomen. \\

Elektronenpaar-Abstossungs-Modell (EPA-Modell): Elektronenpaare stossen sich gegenseitig maximal ab. \\

\subsection{Polare Bindung}
Elektronegativität (EN): Fähigkeit eines Atoms, Bindungselektronen anzuziehen. Die EN ist grösser, je grösser die Rumpfladung und je kleiner der Rumpfradius ist. (im PSE aufgeführt) \\

Polarität einer Bindung: $\Delta EN = | EN_{1} - EN_{2} |$. 
\begin{itemize}
	\item $\Delta EN = 0$: apolare Bindung
	\item $0 < \Delta EN  \leq 1.5$: polare Bindung, Partialladungen $\delta+, \delta-$
	\item $\Delta EN > 1.5$: ionische Bindung
\end{itemize}

Moleküle sind \emph{Dipole} (polare Moleküle), wenn die Schwerpunkte der Partialladungen nicht zusammenfallen. 

\subsection{Zwischenmolekulare Kräfte}
Anziehende Kräfte, die zwischen Molekülen herrschen und Stoffeigenschaften (mp,bp; Mischbarkeit; Viskosität; Oberflächenspannung; ...) beeinflussen. Normalerweise schwächer als kovalente Bindungen.

\begin{itemize}
	\item Dipol-Dipol Kräfte: gegenseitige Anziehung von Dipolmolekülen aufgrund unterschiedlicher Partialladungen. Je polarer, desto grösser.
	\item Van-der-Waals Kräfte: gegenseitige Anziehung von unpolaren Molekülen aufgrund kurzzeitig ungleichmässig verteilter Elektronen. Sehr schwach, mehr $e^-$ $\Rightarrow$ stärker, zwischen allen Molekülen
	\item Wasserstoffbrücken (H-Brücken): Anziehung zwischen stark positiv polarisierten H-Atomen und freien Elektronenpaaren von stark elektronegativen Atomen (F,O,N) $\Rightarrow$ existieren nur bei H-F, H-O oder H-N Bindungen. Stärkste zwischenmolekulare Kräfte.
\end{itemize}

\subsubsection{Siedepunkt abschätzen}
Prinzip: Beim Verdampfen müssen zwischenmolekulare Kräfte überwunden werden $\Rightarrow$ grosse zw.molek. Kräfte $\Leftrightarrow$ hoher bp.
\subsubsection{Lösllichkeit abschätzen}
Prinzip: Ein molekularer Stoff ist löslich, wenn er mit dem Lösungsmittel dieselbe Art zwischenmolekularer Kräfte ausbilden kann. \\
Halbpolare Stoffe besitzen eine polare Gruppe (z.B. -OH) und einen nicht zu langen unpolaren Teil (z.B. KW-Kette). Je länger die KW-Kette desto hydrophober, schlechter löslich wird das Molekül.