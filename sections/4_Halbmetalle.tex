\section{Halbmetalle und Halbleiter}
\textbf{Halbmetalle:} B, Si, Ge, As, Se, Sb, Te, Po, At \\
Halbmetalle zeigen keine einheitlichen Stoffeigenschaften und keinen einheitlichen Aufbau.

\subsection{Silizium}
zweithäufigstes Halbmetall der Erde. kommt nur gebunden vor (in Gestein, Sand oder Bergkristall). $Si$ hat den gleichen Aufbau wie Diamant (Härte 6.5)

\subsection{Halbleiter}
Stoffe mit geringer el. Leitfähigkeit, welche bei steigender Temperatur zunimmt. Halbmetalle sind Halbleiter, aber nicht jeder Halbleiter ist ein Halbmetall. \\
Im Energiebänder-Modell: kleine Verbotene Zone zwischen Valenz- und Leitungsband, ca. 0-3 eV (Si: 1.12 eV bei 300K) \\
Bei steigender Temperatur: Zufuhr von Energie $\Rightarrow$ $e^-$ aus Valenzband kann ins leere Leitungsband springen und hinterlässt Lücke im Valenzband = Defektelektron.

\begin{figure}[htbp]
	\centering
	\includegraphics[width=0.35\linewidth]{images/4_Halbleiter_Energiebaender.png}
\end{figure}

\subsection{Dotierung}
Einbringen von Fremdatomen zur Veränderung der elektrischen Eigenschaften.

\subsubsection{n-Halbleiter}
El. Leitung v.a. durch $e^-$. \\
Bsp. Dotierung von Si mit As: 1 As-Atom pro $10^7$ Si-Atome. $\Rightarrow$ 1 schwach gebundenes Valenz-$e^-$ pro As-Atom $\Rightarrow$ Steigerung der Leitfähigkeit um Faktor $10^6$. \\
Valenz-$e^-$ entspricht einem vollen Energieband (Donatorband) knapp unterhalb des Leitungsbands. 

\subsubsection{p-Halbleiter}
El. Leitung v.a. durch (positive) Defektelektronen. \\
Bsp. Dotierung von Si mit B: 1 B-Atom pro $10^6$ Si-Atome. $\Rightarrow$ 1 fehlendes $e^-$ pro B-Atom $\Rightarrow$ Defektelektron (\emph{Loch}) kann von Si-Valenz-$e^-$ besetzt werden $\Rightarrow$ positive Löcher. \\
Defektelektron entspricht leeren Energieband (Akzeptorband) knapp oberhalb des Valenzbandes.