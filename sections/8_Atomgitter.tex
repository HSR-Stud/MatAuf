\section{Atomgitter}

\subsection{Modifikationen des Kohlenstoffs}

\subsubsection{Diamant}
Im Diamantgitter ($C_D$) ist jedes C-Atom durch Atombindungen an vier andere Kohlenstoffatome gebunden $\Rightarrow$ sehr regelmässiges, stabiles Atomgitter (Tetraeder).\\
Eigenschaften: sehr hart (Härte 10), Zersetzung oberhalb 3600$^\circ$C (beim Abkühlen würde Graphit entstehen), stark lichtbrechend, keine el. Leitfähigkeit, Dichte 3.5 $g/cm^3$

\subsubsection{Graphit}
Graphit ($C_G$) besteht aus Schichten, in denen die C-Atome zu regelmässigen Sechsecken geordnet sind. Jedes C-Atom ist dabei an drei andere C-Atome gebunden. Die C-C Bindungswinkel betragen 120$^\circ$. Jedes C-Atom besitzt noch ein weiteres, nichtbindendes Valenzelektron. Dieses ist über die ganze Schicht frei beweglich (delokalisiert). Zwischen den schichten herrschen VdW-Kräfte ($\Rightarrow$ weich).\\ Eigenschaften:  metallischer Glanz, sehr weich, gute el. Leitfähigkeit, Dichte 2.3 $g/cm^3$, Schmelztemperatur ca. 3700$^\circ$C

\subsubsection{Fullerene}
Fullerene sind kugelförmige Moleküle aus C-Atomen. Das sogenannte Buckminsterfulleren $C_60$ ist das derzeit am Besten erforschte Molekül dieser Art.\\
Eigenschaften: elastisch, z.T. el. leitend, löslich

\subsection{Quarz ($SiO_2$)}
Quarz ist ähnlich aufgebaut wie Diamant. Jedes Si-Atom bildet Bindungen zu vier O-Atomen. Dabei entsteht ein Tetraeder.\\
Eigenschaften: Härte 7, keine el. Leitfähigkeit, Dichte 1.6$g/cm^3$, Schmelztemperatur bei 1700$^\circ$C, Schwingung bei Anlegen eines el.-mag. Feldes

\subsubsection{Piezoeffekt}
Wirken mechanische Druck- oder Zugkräfte entlang bestimmter Kristallachsen auf einen Piezowerkstoff ein, so verschieben sich positiv und negativ geladene Kristallgitterpunkte. Ist das dabei entstehende Summendipolmoment grösser Null, kann es an Aussen angebrachten Elektroden als elektrische Spannung genutzt werden.